\documentclass[12pt]{article}
\usepackage[margin=1in]{geometry}
\usepackage{amsmath, amssymb}
\usepackage{siunitx}
\sisetup{per-mode=symbol,detect-all=true}
\usepackage{setspace}
\onehalfspacing

\begin{document}

\section{Methodology}
We begin by discretizing the crank angle as \(\theta \in [0,2\pi)\) with 360 evenly spaced points. With equal bore diameters, the cross-sectional area is \(A = \pi D^{2}/4\).

\subsection{Step 0: Assisting Calculations}
Before analyzing the cycle, we first compute several derived geometric parameters from the given inputs. We calculate the displacer height as \(h_d = V_d/A\) to determine its physical dimensions. Next, we find the power piston positions at bottom and top dead center using the slider-crank equations, which gives us the swept volume \(V_{\text{swept}} = A(x_{\text{TDC}} - x_{\text{BDC}})\). We then determine the total cylinder volume at BDC using the compression ratio: \(V_{\text{total,BDC}} = V_r - V_d + \mathrm{CR} \cdot V_{\text{swept}}/(\mathrm{CR}-1)\). From this, we compute the crank pin to cylinder roof height as:
\begin{equation}
  H_{\text{tot}} = \frac{V_{\text{total,BDC}}}{A} + h_d + h_{\text{pin}} + \ell_p - r_p,
\end{equation}
where the terms account for the cold/hot space height, displacer height, pin offset, and connecting rod geometry. Finally, we set the regenerator temperature \(T_r = (T_h + T_c)/2\). These calculations establish the complete engine geometry needed for the kinematic and thermodynamic analysis that follows.

\subsection{Step 1: Slider--Crank Kinematics}
Then, to begin the analysis, we calculate piston and displacer positions using slider--crank kinematics with rod obliquity. For a crank radius \(r\) and rod length \(\ell\):
\begin{align}
  \beta(\theta) &= \arcsin\!\biggl( \frac{r}{\ell}\,\sin\theta \biggr),\label{eq:beta-def}\\
  x(\theta) &= \ell\,\cos\beta(\theta) - r\,\cos\theta.
\end{align}
We then apply this to get \(x_{p}(\theta)\) for the power piston and \(x_{d}(\theta+\phi)\) for the displacer (phase shifted by \(\phi\)).

\subsection{Step 2: Cold/Hot Volumes}
Next, we compute the cold and hot space volumes from the piston positions. Cold height is the separation between displacer and power piston; hot height is measured from the top space above the displacer:
\begin{align}
  h_{c}(\theta) &= \bigl[x_{d}(\theta+\phi) - x_{p}(\theta)\bigr] - h_{\mathrm{pin}} - \tfrac{1}{2}h_{d},\\
  h_{h}(\theta) &= H_{\mathrm{tot}} - \tfrac{1}{2}h_{d} - x_{d}(\theta+\phi),
\end{align}
which then gives us volumes
\begin{align}
  V_{c}(\theta) &= A\,h_{c}(\theta), & V_{h}(\theta) &= A\,h_{h}(\theta), & V_{r} &= \text{const}.
\end{align}

\subsection{Step 3: Schmidt Analysis and Mass from BDC}
Now we determine the total gas mass and pressure distribution. We start by setting the total gas mass at a known reference point—bottom dead center (BDC, \(\theta=0\))—using the known absolute pressure \(P_{\mathrm{BDC}}\):
\begin{align}
  m &= \frac{P_{\mathrm{BDC}}}{R}\biggl( \frac{V_{c}(0)}{T_{c}} + \frac{V_{r}}{T_{r}} + \frac{V_{h}(0)}{T_{h}} \biggr).
\end{align}
With this fixed mass, we then calculate the instantaneous absolute pressure at any angle using the Schmidt relation:
\begin{align}
  P(\theta) &= \frac{m\,R}{\dfrac{V_{c}(\theta)}{T_{c}} + \dfrac{V_{r}}{T_{r}} + \dfrac{V_{h}(\theta)}{T_{h}} }.
\end{align}
Finally, we evaluate cycle work numerically as
\begin{equation}
  W = \oint P\,\mathrm{d}V \approx \sum_{k} P(\theta_{k})\,\Delta V(\theta_{k}) \quad \text{(trapezoidal rule)}.
\end{equation}

\subsection{Step 4: Torque with Rod Obliquity}
We then convert pressure to torque. The net axial force on the power piston is \(F_{p}(\theta) = \bigl(P(\theta)-P_{\mathrm{atm}}\bigr)\,A\). Using \(\beta\) from Eq.~\eqref{eq:beta-def} and crank radius \(r_{p}\), the torque becomes:
\begin{equation}
  \tau(\theta) = -\,F_{p}(\theta)\,\frac{r_{p}\,\sin\theta}{\cos\beta(\theta)}.
\end{equation}

\subsection{Step 5: Flywheel Sizing}
Next, we size the flywheel to smooth out torque variations. We start by computing the torque deviation about its mean:
\begin{equation}
  T_{\text{dev}}(\theta) = \tau(\theta) - \overline{\tau}, \quad \text{where } \overline{\tau} = \frac{1}{2\pi}\int_0^{2\pi} \tau(\theta)\,\mathrm{d}\theta.
\end{equation}
We then integrate this deviation to get cumulative energy variation. The energy fluctuation \(\Delta E\) is defined as the peak-to-peak of that signal. This gives us the required inertia:
\begin{equation}
  I_{\mathrm{req}} = \frac{\Delta E}{C_{f}\,\omega_{\!\text{avg}}^{2}}.
\end{equation}
We then model the flywheel as a thick ring (like a washer) with width \(w\), thickness \(t\), density \(\rho\), and outer radius \(R\). Most of the mass is concentrated near the outer edge, giving moment of inertia:
\begin{equation}
  I_{\mathrm{rim}}(R) = \tfrac{1}{2}\,M(R)\,\bigl(R^{2}+R_{\mathrm{in}}^{2}\bigr),\quad M(R)=\rho\,\pi w\,\bigl(R^{2}-R_{\mathrm{in}}^{2}\bigr),\quad R_{\mathrm{in}}=R-t.
\end{equation}
To solve \(I_{\mathrm{rim}}(R)=I_{\mathrm{req}}\), we use a fixed-point iteration. We start from a ring-based guess
\(R^{(0)} = \sqrt{I_{\mathrm{req}}/(\pi\,\rho\,w\,t)} + t/2\). At each step, we evaluate
\(I_{\mathrm{act}} = I_{\mathrm{rim}}\bigl(R^{(k)}\bigr)\), form \(\eta = I_{\mathrm{req}}/I_{\mathrm{act}}\), and rescale:
\begin{equation}
  R^{(k+1)} = R^{(k)}\,\eta^{1/3},\qquad R_{\mathrm{in}}=R^{(k+1)}-t.
\end{equation}
We stop when the relative error \(\lvert I_{\mathrm{act}}-I_{\mathrm{req}}\rvert/I_{\mathrm{req}}\) drops below a tolerance.
Why cube root? For a rim-dominant geometry, inertia scales approximately like \(R^{3}\), so scaling \(R\) by \(\eta^{1/3}\) moves directly toward the required inertia with stable, fast convergence.
Finally, we set diameters \(D_{\mathrm{out}}=2R\), \(D_{\mathrm{in}}=2R_{\mathrm{in}}\) and mass \(M(R)\).

\subsection{Step 6: Energy-Based Dynamics}
Now we simulate how the flywheel affects engine speed. We set the load torque equal to mean engine torque for steady-state operation. The net torque then drives angular acceleration \(\alpha = T_{\text{net}}/I\). Using the work--energy theorem over angle increments with cumulative work \(W_{\text{net}}(\theta)\), we update speed via:
\begin{equation}
  \Omega^{2}(\theta) = \Omega_{\!\text{avg}}^{2} + \frac{2\,W_{\text{net}}(\theta)}{I},\qquad \Omega(\theta) = \sqrt{\Omega^{2}(\theta)}.
\end{equation}
We then normalize \(\Omega\) to recover the target average. This normalization is necessary because the energy-based calculation can drift slightly from the desired mean speed due to numerical integration, so we rescale the entire speed profile to maintain the correct average while preserving the fluctuation pattern.

\subsection{Step 7: Phase Optimization}
Finally, we optimize the phase shift to maximize power output. We perform a three-stage search over \(\phi\): first, a coarse scan from \(30^{\circ}\) to \(150^{\circ}\) in \(2^{\circ}\) steps to find a good neighborhood. Then, we zoom in with a medium scan within \(\pm6^{\circ}\) of the coarse best at \(0.1^{\circ}\) resolution. Finally, we pinpoint the optimum with a fine scan within \(\pm0.5^{\circ}\) at \(0.01^{\circ}\) resolution. At each phase, we compute mean torque integrated over \(\theta\), calculate power as \(\bar{\tau}\,\omega_{\!\text{avg}}\), and select the \(\phi\) that maximizes power.

\end{document}
